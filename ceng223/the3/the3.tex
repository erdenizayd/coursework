\documentclass[12pt]{article}
\usepackage[utf8]{inputenc}
\usepackage[dvips]{graphicx}
\usepackage{epsfig}
\usepackage{fancybox}
\usepackage{verbatim}
\usepackage{array}
\usepackage{latexsym}
\usepackage{alltt}
\usepackage{float}
\usepackage{amsmath}
\usepackage{hyperref}
\usepackage{listings}
\usepackage{color}
\usepackage[hmargin=3cm,vmargin=5.0cm]{geometry}
\topmargin=-1.8cm
\addtolength{\textheight}{6.5cm}
\addtolength{\textwidth}{2.0cm}
\setlength{\oddsidemargin}{0.0cm}
\setlength{\evensidemargin}{0.0cm}

\newcommand{\HRule}{\rule{\linewidth}{1mm}}
\newcommand{\kutu}[2]{\framebox[#1mm]{\rule[-2mm]{0mm}{#2mm}}}
\newcommand{\gap}{ \\[1mm] }

\newcommand{\Q}{\raisebox{1.7pt}{$\scriptstyle\bigcirc$}}

\lstset{
    %backgroundcolor=\color{lbcolor},
    tabsize=2,
    language=C++,
    basicstyle=\footnotesize,
    numberstyle=\footnotesize,
    aboveskip={0.0\baselineskip},
    belowskip={0.0\baselineskip},
    columns=fixed,
    showstringspaces=false,
    breaklines=true,
    prebreak=\raisebox{0ex}[0ex][0ex]{\ensuremath{\hookleftarrow}},
    %frame=single,
    showtabs=false,
    showspaces=false,
    showstringspaces=false,
    identifierstyle=\ttfamily,
    keywordstyle=\color[rgb]{0,0,1},
    commentstyle=\color[rgb]{0.133,0.545,0.133},
    stringstyle=\color[rgb]{0.627,0.126,0.941},
}


\begin{document}



\noindent
\HRule \\[3mm]
\small
\begin{tabular}[b]{lp{3.8cm}r}
{} Middle East Technical University &  &
{} Department of Computer Engineering \\
\end{tabular} \\
\begin{center}

                 \LARGE \textbf{CENG 223} \\[4mm]
                 \Large Discrete Computational Structures \\[4mm]
                \normalsize Fall '2020-2021 \\
                    \Large Homework 3 \\
                \normalsize Student Name and Surname: Mehmet Erdeniz Aydoğdu \\
                \normalsize Student Number:  2380103\\
\end{center}
\HRule


\section*{Question 1}
The congruence's left side can be written as $(2^2)^{11} + (4^4)^{11} + (6^6)^{11} + (8^8)^{10} + (10^{11})^{10}\:mod\:11$ \newline
Fermat's little theorem states that: \newline \newline
$a^{p-1} \equiv 1\:(mod\:p)$ if a is not divisible by p, also,\newline
$a^p \equiv a\:(mod\:p)$ for integer a and prime p \newline\newline
$(2^2)^{11} \equiv 4\:(mod\:11)$ \newline
$(4^4)^{11} \equiv 256\:(mod\:11) \equiv 3\:(mod\:11)$ \newline
$(6^6)^{11} \equiv 46.656\:(mod\:11) \equiv 5\:(mod\:11)$ \newline
$(8^8)^{10} \equiv 1\:(mod\:11)$ \newline
$(10^{11})^{10} \equiv 1\:(mod\:11)$ Thus, \newline \newline
$2^{22}\:+\:4^{44}\:+\:6^{60}\:+\:8^{80}\:+\:10^{110}\:mod\:11 \equiv 4\:+\:3\:+\:5\:+\:1\:+\:1\:mod\:11 \equiv 14\:mod\:11 \equiv 3$
\section*{Question 2}
7n + 4 = (5n + 3) x 1 + (2n + 1) \newline
5n + 3 = (2n + 1) x 2 + (n + 1) \newline
2n + 1 = (n + 1) x 1 + n \newline
n + 1 = n x 1 + 1 \newline
n = n x 1 \newline
The last non-zero remainder is 1. Thus, gcd(5n + 3, 7n + 4) = 1 by Euclidian Algorithm.
\section*{Question 3}
$m^2\:=\:n^2\:+kx$, add -($n^2$) to each side \newline
$m^2\:-\:n^2\:=kx$, as $m^2\:-\:n^2\:=\:(m\:-\:n)(m\:+\:n)$ \newline
$\frac{(m\:-\:n)(m\:+\:n)}{x} = k$ and k is an integer, thus x divides (m + n) or (m - n)

\section*{Question 4}
Say f(n) = 1 + 4 + 7 + ... + (3n - 2) = $\frac{n(3n - 1)}{2}$\newline
For the first step, we need to prove f(1), which is $\frac{1(3-1)}{2} = 1$\newline 
For inductive step, assume f(k) is true. Then, f(k) = $\frac{k(3k-1)}{2}$ is true. \newline
Then, f(k+1) = 1 + 4 + 7 + ... + (3k - 2) + (3k + 1) = $\frac{k(3k-1)}{2}$ + (3k + 1) \newline
$\frac{k(3k-1)}{2}$ + (3k + 1) = $\frac{3k^2+5k+2}{2}$, as $3k^2+5k+2 = (3k + 1)(k + 1)$ \newline
Thus, f(k+1) is true. Then, f(n) is true.

\end{document}

