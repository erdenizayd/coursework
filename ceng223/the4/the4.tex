\documentclass[11pt]{article}
\usepackage[utf8]{inputenc}
\usepackage{float}
\usepackage{amsmath}
\usepackage{amssymb}

\usepackage[hmargin=3cm,vmargin=6.0cm]{geometry}
%\topmargin=0cm
\topmargin=-2cm
\addtolength{\textheight}{6.5cm}
\addtolength{\textwidth}{2.0cm}
%\setlength{\leftmargin}{-5cm}
\setlength{\oddsidemargin}{0.0cm}
\setlength{\evensidemargin}{0.0cm}

% symbol commands for the curious
\newcommand{\setZp}{\mathbb{Z}^+}
\newcommand{\setR}{\mathbb{R}}
\newcommand{\calT}{\mathcal{T}}

\begin{document}

\section*{Student Information } 
%Write your full name and id number between the colon and newline
%Put one empty space character after colon and before newline
Full Name :  Mehmet Erdeniz Aydoğdu \\ 
Id Number :  2380103 \\ 

% Write your answers below the section tags
\section*{Answer 1}
When we choose 1 star, 2 habitable planets and 8 non-habitable planets, the number of combinations we can get can be counted by: \newline \newline
\centerline{$\binom{10}{1} \binom{20}{2} \binom{80}{8}$} \newline \newline
Then, the number of ways for the case of 6 non-habitable planets between 2 habitable planets can be counted by \newline \newline
\centerline{$\binom{10}{1} \binom{20}{2} \binom{80}{8} (\binom{8}{6}\:6!\:2!\:2!)$} \newline \newline
The number of ways for the case of 7 non-habitable planets between habitable ones is: \newline \newline
\centerline{$\binom{10}{1} \binom{20}{2} \binom{80}{8} (\binom{8}{7}\:7!\:2!\:1!)$} \newline \newline
Lastly, for 8 non-habitable planets between habitable ones, the number of ways can be counted by: \newline \newline
\centerline{$\binom{10}{1} \binom{20}{2} \binom{80}{8} (\binom{8}{8}\:8!\:2!\:0!)$} \newline \newline
By the rule of sum, the number of ways are: \newline \newline
\centerline{$\binom{10}{1} \binom{20}{2} \binom{80}{8} (\binom{8}{6}\:6!\:2!\:2!\:+\:\binom{8}{7}\:7!\:2!\:1!\:+\:\binom{8}{8}\:8!\:2!\:0!)$}
\newpage
\section*{Answer 2}
Here, the characteristic equation of the relation is $r^3 - 2r^2 - 15r + 36 = 0$ \newline
Using this equation we can derive $(r-3)^3(r+4) = 0$ \newline
Solving this equation, we get 2 roots $r_1 = 3$ with multiplicity 3 and $r_2 = -4$ \newline
Thus, homogeneous solution is \newline \newline
\centerline{$a^{(h)}_n = (A_1n^2 + A_2n + A_3)3^n + B(-4)^n$} \newline \newline
where $A_1,\:A_2,\:A_3$ and $B$ are arbitrary constants.\newline \newline
Since particular part only have constant coefficient "1", $a^{(p)}_n$ is of type $P 2^n$ \newline
We get $(P\:-\:2P\:-\:15P\:+\:36P)2^n = 2^n$ by substituting the particular solution to the equation. Thus, \newline \newline
$P\:-\:2P\:-\:15P\:+\:36P\:=\:20P\:=\:1$, then $P\:=\:\frac{1}{20}$\newline \newline
\centerline{$a^{(p)}_n = \frac{1}{20}2^n$}
\section*{Answer 3}
In order to generate the recurrence relation, we have two possibilities: 
\begin{enumerate}
    \item For an n-digit valid code $a_n$, adding an odd digit to a (n-1)-digit invalid code $a_{n-1}$. There are total number of $10^{n-1}$ codes, thus $10^{n-1} - a_{n-1}$ number of invalid codes. Because there are 5 odd digits, we have $5(10^{n-1} - a_{n-1})$ valid codes.
    \item For an n-digit valid code $a_n$, adding an even digit to a (n-1)-digit valid code $a_{n-1}$. Since there are five even digits, we have $5a_{n-1}$
\end{enumerate}
    For n = 1, we only have 1-digit codes, which contain odd digits, of number five. Then, $a_1 = 5$ Thus, the recurrence relation is:\newline \newline
    \centerline{$a_n = 5a_{n-1} + 5(10^{n-1} - a_{n-1}) = 5 \times 10^{n-1}$}
\newpage
\section*{Answer 4}
Writing the equation in summation form: \newline \newline
\centerline{$\sum_{k=3}^{\infty}a_k x^k = 3\sum_{k=3}^{\infty}a_{k-1}x^k - 3\sum_{k=3}^{\infty}a_{k-2}x^k + \sum_{k=3}^{\infty} a_{k-3}x^k$} \newline \newline
By making adjustments on the right hand side: \newline \newline
\centerline{$\sum_{k=3}^{\infty}a_k x^k = 3x\sum_{k=3}^{\infty}a_{k-1}x^{k-1} - 3x^2\sum_{k=3}^{\infty}a_{k-2}x^{k-2} + x^3\sum_{k=3}^{\infty} a_{k-3}x^{k-3}$} \newline \newline
For $G(x) = \sum_{k=0}^{\infty}a_k x^k$, writing the equation using $G(x)$: \newline \newline
\centerline{$G(x) - a_0 - a_1x - a_2x^2 = 3x(G(x) - a_0 - a_1x) - 3x^2(G(x) * a_0) + x^3G(x)$} \newline \newline
Substitute $a_0 = 1$, $a_1 = 3$, $a_2 = 6$ and reorganize the equation: \newline \newline
\centerline{$G(x)(1 - 3x + 3x^2 - x^3) = 1 + 3x + 6x^2 - 3x - 9x^2 + 3x^2$} \newline \newline
\centerline{$G(x)(1-x)^3 = 1$, thus, $G(x) = \frac{1}{(1-x)^3}$} \newline \newline
By the 9th generating function in the table, $a_k = C(k+2,k) = \frac{(k+1)(k+2)}{2}$ 
\section*{Answer 5}
\paragraph{a.}
\begin{enumerate}
    \item We need to show that R is reflexive: \newline
    ((a,b),(a,b)) $\in$ R iff a + b = b + a, so R is reflexive.
    \item To show R is symmetric: \newline
    ((a,b),(c,d)) $\in$ R iff a + d = b + c \newline
    ((c,d),(a,b)) $\in$ R iff c + b = d + a, so R is symmetric.
    \item To show R is transitive: \newline
    ((a,b),(c,d)) $\in$ R iff a + d = b + c \newline
    ((c,d),(e,f)) $\in$ R iff c + f = d + e \newline
    Using first and second equation, we get a + f = b + e, which is \newline
    ((a,b),(e,f)) $\in$ R iff a + f = b + e, so R is transitive.
\end{enumerate}
Thus, R is a equivalence relation.
\paragraph{b.}
((a,b),(1,2)) $\in$ R iff a + 2 = b + 1 \newline \newline
From the equation, we get b = a + 1, then \newline \newline
$[(1,2)]_R = {(a,a + 1)}$

\end{document}